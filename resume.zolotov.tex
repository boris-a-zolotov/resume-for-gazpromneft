\documentclass[11pt,a4paper]{article}
\usepackage{notes,wrapfig}
\pagestyle{empty} \textheight=28cm

\definecolor{hlink}{rgb}{0,0,65}
\usepackage[colorlinks=true,urlcolor=hlink]{hyperref}

\renewcommand{\thesection}{\arabic{section}.\!}
\def\sec#1{\section{#1} \vspace{-0.22cm}}

\begin{document}
\ \vspace{-1.5cm}

%%%%%%%%%%%%%%%%
%%%%%%%%%%%%%%%%

\begin{center}
{\LARGE\bf Резюме для проекта НТЦ\,«Газпром—Нефть»} \\ \smallskip
{\large\it Борис Золотов, студент СПбГУ, группа 15.Б-02мм}
\end{center}\vspace{1mm}\hrule\vspace{2mm}



\sec{Личная информация}
\noindent Дата рождения: 27.\,11.\,1997 (20 лет).

\ms Текущее место работы: студент III курса Мат-меха СПбГУ\scolon преподаватель математического кружка ЛНМО\scolon организатор \href{http://mathnonstop.ru/}{олимпиады «Математика НОН-СТОП»}.



\sec{Желаемая должность}
\noindent Технический писатель научного проекта.



\sec{Мотивация}
\noindent Я был бы искренне рад помочь научному проекту в его самой кропотливой и нацеленной на аудиторию «архивной» части, принять участие в создании публикуемых документов\scolon погрузиться в атмосферу научного исследования, познающего планету, на которой мы живём, и полезного для обеспечения человечества энергией и теплом.



\sec{Профессиональные навыки}
\def\ite{\item[—]}

\begin{itemize}
\setlength\itemsep{0.2ex}

\ite Офисный пакет {\sffamily LibreOffice} (открытый аналог {\sffamily Microsoft Office}) версий 4–6: программы {\sffamily LibreOffice} {\sffamily Writer}, {\sffamily Calc}, {\sffamily Impress} (для создания документов, таблиц, презентаций)\scolon

\ite Система вёрстки \LaTeX: работа с изображениями, таблицами (в том числе пакет {\sffamily makecell}), многострочными формулами, коммутативными диаграммами, презентациями (пакет {\sffamily beamer}), гиперссылками\scolon документами, состоящими из многих частей ({\tt\textbackslash input}), исходным кодом в теле документа\scolon

\ite Программы {\sffamily Inkscape} и {\sffamily Adobe Photoshop} для создания и обработки внешнего вида документов\scolon

\ite Редактор {\sffamily emacs}, позволяющий вставлять формулы и математические символы в «обычные» тексты наподобие сообщений или электронных писем\scolon

\ite Базовое знание языка разметки {\tt xml}, в частности формата {\sffamily GPX}. Языки програмирования: {\tt Objective Caml}, {\tt Free Pascal}, {\tt Haskell}\scolon

\ite ЕГЭ по русскому языку — 98 баллов. Английский язык — сертификат уровня B2.
\end{itemize}



\sec{Опыт работы}
\noindent Принимал участие в работе следующих проектов:\vspace{-0.18cm}
\begin{itemize}
\setlength\itemsep{0.2ex}

\ite Электронный конспект семинаров по алгебре (группа К.И.\,Пименова), I–III семестр. Результат — $\approx\!70$--страничный математический \LaTeX--документ в каждом из се-\linebreak местров. Руководитель — П.С.\,Перстнева.

\ite Научная работа по теме «Комбинаторика слов», 2015: была представлена на конференциях BalticSEF, Intel ISEF, ICYS\scolon опубликована в {\sffamily ar$\chi$iv}. Руководитель — С.И.\linebreak Кублановский.

\ite Создание условий олимпиады «Математика НОН-СТОП», 2016–2018: более 70 задач для школьников в 7 вариантах. Руководитель — И.А.\,Чистяков.

\ite Проверка олимпиады «Математика НОН-СТОП»: работа с таблицами с большим количеством данных и формул.
\end{itemize}

\noindent Из менее важных документов, над которыми работал: рефераты, решения задач, дипломы ПОМИ и Балтийского научно-инженерного конкурса.



\vfill\eject
\ \\
\sec{Примеры работ}
\ 
\begin{itemize}

\ite Курсовая работа, весенний семестр 2018:

\begin{quote}
\url{https://github.com/boris-a-zolotov/Term-project-TeX-layout}
\end{quote}

\ite Электронный конспект по алгебре, II семестр:

\begin{quote}
\url{http://bit.ly/algebra_praxis_ii}
\end{quote}

\ite «Another Solution to the Thue Problem of Non-Repeating Words», апрель 2015:

\begin{quote}
\url{https://arxiv.org/pdf/1505.00019.pdf}
\end{quote}

\ite Условия олимпиады «Математика НОН-СТОП», 2018:

\begin{quote}
\url{http://bit.ly/nonstahp2018}
\end{quote}

\ite «Реформы в истории России», декабрь 2015:

\begin{quote}
\url{https://yadi.sk/i/NFCBR76u3Tkp9E}
\end{quote}

\ite Награда ПОМИ РАН, 2018:

\begin{quote}
\url{http://bit.ly/zolotov_pdmi_png}
\end{quote}
\end{itemize}


\end{document}